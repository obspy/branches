\documentclass[handout]{beamer}
\definecolor{lmu@green}{rgb}{0,0.58,0.25} % use structure theme to change
\definecolor{lmu@darkgreen}{rgb}{0,0.4,0.12} % use structure theme to change
% Uncomment the following for handouts.
%\documentclass[handout]{beamer}
\usepackage[utf8x]{inputenc}
\usepackage{amsmath,amsfonts,amssymb}
\setbeamertemplate{navigation symbols}{}

% Python listing setup

\usepackage{color}
\usepackage[procnames]{listings}
%\usepackage{textcomp}
%\usepackage{setspace}
\usepackage[]{xcolor}
%\renewcommand{\lstlistlistingname}{Code Listings}
%\renewcommand{\lstlistingname}{Code Listing}
\definecolor{gray}{gray}{0.5}
\definecolor{green}{rgb}{0,0.5,0}
\definecolor{lightgreen}{rgb}{0,0.7,0}
\definecolor{purple}{rgb}{0.5,0,0.5}
\definecolor{darkred}{rgb}{0.7,0,0}

\usepackage[T1]{fontenc}
\usepackage{lmodern}
\usepackage[scaled]{beramono}

\lstnewenvironment{python}[1][]{
\lstset{
% Escape with funnyeyes.
escapeinside={(*@}{@*)},
language=python,
basicstyle=\ttfamily\small,
stringstyle=\color{green},
showstringspaces=false,
alsoletter={1234567890},
otherkeywords={\ , \}, \{},
keywordstyle=\color{blue},
emph={access,and,as,break,class,continue,def,del,elif,else,%
except,exec,finally,for,from,global,if,import,in, is,%
lambda,not,or,pass,print,raise,return,try,while,assert},
emphstyle=\color{orange}\bfseries,
emph={[2]self},
emphstyle=[2]\color{gray},
emph={[4]ArithmeticError,AssertionError,AttributeError,BaseException,%
DeprecationWarning,EOFError,Ellipsis,EnvironmentError,Exception,%
False,FloatingPointError,FutureWarning,GeneratorExit,IOError,%
ImportError,ImportWarning,IndentationError,IndexError,KeyError,%
KeyboardInterrupt,LookupError,MemoryError,NameError,None,%
NotImplemented,NotImplementedError,OSError,OverflowError,%
PendingDeprecationWarning,ReferenceError,RuntimeError,RuntimeWarning,%
StandardError,StopIteration,SyntaxError,SyntaxWarning,SystemError,%
SystemExit,TabError,True,TypeError,UnboundLocalError,UnicodeDecodeError,%
UnicodeEncodeError,UnicodeError,UnicodeTranslateError,UnicodeWarning,%
UserWarning,ValueError,Warning,ZeroDivisionError,abs,all,any,apply,%
basestring,bool,buffer,callable,chr,classmethod,cmp,coerce,compile,%
complex,copyright,credits,delattr,dict,dir,divmod,enumerate,eval,%
execfile,exit,file,filter,float,frozenset,getattr,globals,hasattr,%
hash,help,hex,id,input,int,intern,isinstance,issubclass,iter,len,%
license,list,locals,long,map,max,min,object,oct,open,ord,pow,property,%
quit,range,raw_input,reduce,reload,repr,reversed,round,set,setattr,%
slice,sorted,staticmethod,str,sum,super,tuple,type,unichr,unicode,%
vars,xrange,zip},
emphstyle=[4]\color{purple}\bfseries,
upquote=true,
morecomment=[s][\color{lightgreen}]{"""}{"""},
commentstyle=\color{red}\slshape,
literate={>>>}{\bfseries{\textcolor{darkred}{>{>}>}}}3%
         {...}{{\textcolor{gray}{...}}}3,
procnamekeys={def,class},
procnamestyle=\color{blue}\textbf,
framexleftmargin=1mm, framextopmargin=1mm,
rulesepcolor=\color{blue},#1
}}{}



\usetheme{LMU}
\usecolortheme{lmu}
\useinnertheme{lmu}
\useoutertheme{lmu}

% -----------------------------------------------------------------------------
%
%\newtheorem{definition}{Definition}
\newcommand{\foot}[1]{_{\mbox{\footnotesize #1}}}
\newcommand{\head}[1]{^{\mbox{\footnotesize #1}}}
%
%
\newcommand{\ones}{\mathbb{I}}
\newcommand{\nat}{\mathbb{N}}
\newcommand{\real}{\mathbb{R}}
\newcommand{\ganz}{\mathbb{Z}}
%
%
\newcommand{\RRE}{\mbox{RRE}}
\newcommand{\nnz}[1]{\mbox{nnz}(#1)}
\newlength{\Hoehe}
\renewcommand{\vec}[1]{#1}
\newlength{\GLaenge}
\setlength{\GLaenge}{3.5cm}
%
%
\definecolor{MyGrey}{gray}{0.45}
\def\bstheta{\boldsymbol{\theta}}
\def\bsalpha{\boldsymbol{\alpha}}
\def\bsk{\boldsymbol{k}}
\def\bsx{\boldsymbol{x}}
\def\bsh{\boldsymbol{h}}
%
% Centred minipage environment
%
\newenvironment{cmpage}[1]{
\begin{center}
\begin{minipage}{#1\textwidth}}%
{\end{minipage}\end{center}}
%
%
\newcommand{\POS}{\color{blue}\item [\boldmath{$+$}]}
\newcommand{\NEG}{\color{red}\item [{\boldmath$-$}]}
\newcommand{\NTR}{\color{black}\item [$\circ$]}
\newcommand{\f}[1]{\mathfrak{#1}}
\newcommand{\old}{^{\mbox{\small \color{blue} old}}}
\newcommand{\new}{^{\mbox{\small \color{red} new}}}
\newcommand{\diag}[1]{\mbox{diag}\left(#1\right)}
%
%
\newcommand{\myBlank}{\textvisiblespace}
\newcommand{\noSpace}{\makebox[0pt]{\quad}}
%
% Old style colour commands
%
\newcommand{\CB}{\color{blue}}
\newcommand{\CR}{\color{red}}
\newcommand{\CG}{\color{green}}
\newcommand{\CC}{\color{cyan}}
%
\definecolor{myWhite}{rgb}{1.00,1.00,1.00}  % real white
\definecolor{myGrey}{rgb}{0.78,0.83,0.94}   % 'light grey blue'
\definecolor{myYellow}{rgb}{1.00,1.00,0.00} % yellow
\definecolor{myOrange}{rgb}{1.00,0.65,0.00} % orange
\definecolor{myCyan}{rgb}{0.00,1.00,1.00}   % cyan
%
% Some abbrevs for setting brief code parts
%
\newcommand{\ttA}{\mbox{\texttt{A}}}
\newcommand{\ttB}{\mbox{\texttt{B}}}
\newcommand{\ttC}{\mbox{\texttt{C}}}
\newcommand{\ttD}{\mbox{\texttt{D}}}
\newcommand{\code}[1]{\mbox{\texttt{#1}}}
\newcommand{\ccode}[1]{\cemphd{\texttt{#1}}}
%
% Commands for slides taken from 'Insides'
%
\newcommand{\rst}{\textcolor{emphcolora}{\ast}}
\newcommand{\bst}{\textcolor{emphcolorb}{\ast}}
%
%
%
\definecolor{textcolor} {rgb}{0,0,0}
\definecolor{decocolor} {rgb}{0,0,0}
\definecolor{emphcolora}{rgb}{1,0,0}              % pure red
\definecolor{emphcolorb}{rgb}{0,0,1}              % pure blue
\definecolor{emphcolorc}{cmyk}{0,1,0,0}           % pure magenta
%\definecolor{emphcolord}{cmyk}{0.64,0,0.95,0.20} % sort of green
\definecolor{emphcolord}{rgb}{0,0.4,0.12}         % same as lmu@darkgreen
\definecolor{emphcolore}{cmyk}{1,0,0,0}           % pure cyan
\definecolor{linkcolor} {rgb}{0,0,0}
%
% Commands emphasising text using color
%
\newcommand{\cempha}[1]{{\color{emphcolora}#1}}
\newcommand{\cemphb}[1]{{\color{emphcolorb}#1}}
\newcommand{\cemphc}[1]{{\color{emphcolorc}#1}}
\newcommand{\cemphd}[1]{{\color{emphcolord}#1}}
\newcommand{\cemphe}[1]{{\color{emphcolore}#1}}
\newcommand{\cemphf}[1]{{\color{decocolor}#1}}
% -----------------------------------------------------------------------------
% myColorBox
% -----------------------------------------------------------------------------
\setbeamercolor{myBoxColor}{fg=black,bg=white}
\setbeamercolor{myBoxColorHead}{fg=red,bg=white}
% \newenvironment{myColorBox}[2]{%
% \begin{beamerboxesrounded}[shadow=true,lower=myBoxColor,upper=myBoxColorHead,
% width=#1\textwidth]{#2}}%
% {\end{beamerboxesrounded}}
\newenvironment{myColorBox}[2]{%
\begin{cmpage}{#1}%
\begin{beamerboxesrounded}[shadow=true,lower=myBoxColor,upper=myBoxColorHead]%
{#2}}%
{\end{beamerboxesrounded}\end{cmpage}}
%
% -----------------------------------------------------------------------------
% Math Operators, alternate greek symbols and the like
% -----------------------------------------------------------------------------
\DeclareMathOperator{\grad}{grad}
\DeclareMathOperator{\mydiv}{div}
\DeclareMathOperator{\Grad}{grad}
\DeclareMathOperator{\Div}{div}
%\newcommand{\grad}{\mbox{grad}}
%\newcommand{\mydiv}{\mbox{div}}
\renewcommand{\rho}{\varrho}
%
% -----------------------------------------------------------------------------
% Some color defintions to be compatible with XFIG
% -----------------------------------------------------------------------------
%
\definecolor{XFIGgold}{rgb}{1.00,0.84,0.00}
\definecolor{XFIGltblue}{rgb}{0.53,0.81,1.00}
\definecolor{XFIGred}{rgb}{1.00,0.00,0.00}
% -----------------------------------------------------------------------------

\usepackage{lmodern}

% Meta information.
\title{ObsPy: A powerful instrument for seismological data software development}
\subtitle{ObsPy Workshop}
\author{ObsPy Developers}
\date{Zurich, Sept 6-8 2012}
\institute{SED/ETHZ}

\title{ObsPy: Station and Event Data Handling}
\begin{document}

\frame[plain]{\titlepage}



\begin{frame}[fragile, plain]{Inventory Data - obspy.xseed}
    \begin{itemize}
        \item Can currently read/write/convert between SEED and XML-SEED.
        \item RESP file support.
        \item StationXML support is planned.
    \end{itemize}


\footnotesize
\begin{myColorBox}{0.95}{}
\begin{semiverbatim}
000001V 010009402.3121970,001,00:00:00.0000~2038,001,00:00:00.0000~
2009,037,04:32:41.0000~BayernNetz~~0110032002RJOB 000003RJOB 000008
...
\end{semiverbatim}
\end{myColorBox}

\large
\begin{center}
    $\Updownarrow$
\end{center}

\footnotesize


\begin{myColorBox}{0.95}{}
\begin{semiverbatim}
<?xml version='1.0' encoding='utf-8'?>
<xseed version="1.0">
  <volume_index_control_header>
    <volume_identifier blockette="010">
      <version_of_format>2.4</version_of_format>
      <logical_record_length>12</logical_record_length>
      <beginning_time>1970-01-01T00:00:00</beginning_time>
      <end_time>2038-01-01T00:00:00</end_time>
...
\end{semiverbatim}
\end{myColorBox}

\normalsize

\end{frame}


\begin{frame}[fragile, plain]{obspy.xseed - Example usage}
\begin{myColorBox}{0.95}{}
\begin{python}
>>> from obspy.xseed import Parser
>>> p = Parser("dataless_SEED")
>>> print p
BW.FURT..EHZ | 2001-01-01T00:00:00.000000Z -
BW.FURT..EHN | 2001-01-01T00:00:00.000000Z -
BW.FURT..EHE | 2001-01-01T00:00:00.000000Z -
>>> p.getCoordinates("BW.FURT..EHZ")
{"elevation": 565.0, "latitude": 48.162899,
 "longitude": 11.2752}
>>> p.getPAZ("BW.FURT..EHZ")
{"digitizer_gain": 1677850.0,
 "gain": 1.0,
 "poles": [(-4.444+4.444j), (-4.444-4.444j), (-1.083+0j)],
 "seismometer_gain": 400.0,
 "sensitivity": 671140000.0,
 "zeros": [0j, 0j, 0j]}
\end{python}
\end{myColorBox}
\end{frame}


\begin{frame}[fragile, plain]{obspy.xseed - Example usage}
\begin{myColorBox}{0.95}{}
\begin{python}
>>> p.writeXSEED("dataless.xml")
# Edit it ...
>>> p = Parser("dataless.xml")
>>> p.writeSEED("edit_dataless_SEED")
\end{python}
\end{myColorBox}
\end{frame}


\begin{frame}[fragile, plain]{obspy.xseed - Exercise}
    \begin{itemize}
        \item Read the \textbf{BW.FURT..EHZ.D.2010.005} waveform example file.
        \item Cut out some minutes of interest.
        \item Read the \textbf{dataless.seed.BW\_FURT} SEED file.
        \item Correct the trimmed waveform file with the poles and zeros from
            the dataless SEED file using \textit{st.simulate()}. This will,
            according to the SEED convention, correct to $m/s$.
        \item (Optional) Read the file again and convert to $m$ by adding an
            extra zero. Choose a sensible waterlevel.
        \item (Optional) Convert the SEED file to XSEED, edit some values and
            convert it back to SEED again. This requires some knowledge of the
            general SEED file structure.
    \end{itemize}
\end{frame}


\begin{frame}[plain, fragile]{Events - Work in progress}
    \begin{itemize}
        \item Aims to get a unified interface with read and write support independent of the data source, similar to how the Stream and Trace classes handle waveform data.
        \item Currently only supports QuakeML and is modelled after it.
    \end{itemize}
\begin{myColorBox}{0.95}{}
\begin{python}
>>> from obspy.core.events import readEvents
>>> url = "http://www.seismicportal.eu/services/..."
>>> catalog = readEvents(url)
>>> print catalog
99 Event(s) in Catalog:
2012-04-11T10:43:09.400000Z |  ... | 8.2 Mw | ...
2012-04-11T08:38:33.000000Z |  ... | 8.4 M  | ...
...
\end{python}
\end{myColorBox}
\end{frame}

\begin{frame}[plain, fragile]{Events - Basic Structure}
    \begin{itemize}
        \item The \textbf{readEvents()} function always returns a
            \textbf{Catalog} object, which is a collection of \textbf{Event}
            objects.
    \end{itemize}
\begin{myColorBox}{0.95}{}
\begin{python}
>>> from obspy.core.events import readEvents
>>> cat = readEvents()
>>> type(cat)
obspy.core.event.Catalog
>>> type(cat[0])
obspy.core.event.Event
\end{python}
\end{myColorBox}
\end{frame}

\begin{frame}[plain, fragile]{Events - Basic Structure}
\begin{myColorBox}{0.95}{}
\begin{python}
>>> event = cat[0]
>>> print event
Event:  2012-04-04T14:...| +41.818,  +79.689 | 4.4 mb

           resource_id: ResourceIdentifier(...)
            event_type: "not reported"
         creation_info: CreationInfo
            agency_uri: ResourceIdentifier(...)
            author_uri: ResourceIdentifier(...)
         creation_time: UTCDateTime(2012, 4, 4, 16, 40, 50)
               version: "1.0.1"
        ---------
                    origins: 1 Elements
                 magnitudes: 1 Elements
\end{python}
\end{myColorBox}
\end{frame}

\begin{frame}[plain, fragile]{Events - Basic Structure}
    \begin{itemize}
        \item \textbf{Event} objects are again collections of other resources.
    \end{itemize}
\begin{myColorBox}{0.95}{}
\begin{python}
>>> type(event.origins[0])
obspy.core.event.Origin
>>> type(event.magnitudes[0])
obspy.core.event.Magnitude
>>> print event.origins[0]
Origin
         resource_id: ResourceIdentifier(...)
                time: UTCDateTime(...)
            latitude: 41.818
           longitude: 79.689
               depth: 1.0
          depth_type: "from location"
           method_id: ResourceIdentifier(...)
  used_station_count: 16
       azimuthal_gap: 231.0
       ...
\end{python}
\end{myColorBox}
\end{frame}

\begin{frame}[plain, fragile]{Events - Resource References}
    \begin{itemize}
        \item In QuakeML resources can refer to each other using a unique identifier string.
        \item These connections are preserved in obspy.core.event.
        \item This works across file boundaries assuming all necessary resources have been read before.
    \end{itemize}
\begin{myColorBox}{0.95}{}
\begin{python}
>>> magnitude = event.magnitudes[0]
# Retrieve the associated Origin object.
>>> print magnitude.origin_id
quakeml:eu.emsc/origin/rts/261020/782484
>>> origin = magnitude.origin_id.getReferredObject()
>>> print origin
Origin
  resource_id: ResourceIdentifier(...)
         time: UTCDateTime(2012, 4, 4, 14, 21, 42, 300000)
     latitude: 41.818
    longitude: 79.689
    ...
\end{python}
\end{myColorBox}
\end{frame}

\begin{frame}[plain, fragile]{The Catalog object}
    \begin{itemize}
        \item The Catalog object contains some convenience methods to make working with events easier.
        \item Events can be filtered with various keys.
    \end{itemize}
\begin{myColorBox}{0.95}{}
\begin{python}
>>> small_magnitude_events = cat.filter("magnitude <= 4.0")
\end{python}
\end{myColorBox}

    \begin{itemize}
        \item They can be plotted using the basemap module.
    \end{itemize}
\begin{myColorBox}{0.95}{}
\begin{python}
    >>> cat.plot()
\end{python}
\end{myColorBox}
    \begin{itemize}
        \item And they can be written.
    \end{itemize}
\begin{myColorBox}{0.95}{}
\begin{python}
>>> cat.write("modified_events.xml", format="quakeml")
\end{python}
\end{myColorBox}
\end{frame}

\begin{frame}[fragile, plain]{obspy.core.event - Exercise}
    \begin{itemize}
        \item Get the 5 latest events, directly from the seimicportal.eu webservice.
            \begin{itemize}
                \item \footnotesize{\textbf{cat = readEvents(\\
                 "http://www.seismicportal.eu/services/event/latest?num=5")}}
            \end{itemize}
        \item Plot the events.
        \item Print the resulting Catalog object and filter it, so it only contains the three largest events.
        \item Now assume you did a new magnitude estimation and want to add it
            to one event. Create a new magnitude object, fill it with some
            values and append it to magnitude list of the largest event.
        \item Write the Catalog as a QuakeML object.
    \end{itemize}
\end{frame}



\end{document}
